\section{Part I: Option pricing using interest rate trees}

%% options and interest rate trees
\begin{frame}
  \frametitle{What are options?}

Types of options: Call and put options.

(The following will focus on european call options, but the
approach we present applies to put options as well.)

Informally: Options are the right but not the obligation to
buy some asset for some price at expiration date.

Formally:

Given the following:
%
\begin{itemize}
  \item Price of underlying asset: $S$
  \item Strike price: $K$
  \item Maturity date: $T$
\end{itemize}
%
Then an option is a contract between two parties, such that
the buyer has the right to buy the underlying asset at price
$K$ when the contract expires at time $T$. The buyers payoff
at time $T$ can be described as:
%
\begin{equation}
  \operatorname{max} \{ 0, S - K \}
\end{equation}
%
\end{frame}

\begin{frame}
  \frametitle{What are interest rates?}
Include:
%
\begin{itemize}
  \item current market term structure (yields on bonds?)
  \item the short rate tree
  \item branching process (exhibit 1, 2, 3)
  \item alphas, qs
  \item the tree construction two stage process: 1) compute
    simple tree, 2) introduce time-varying drift.
\end{itemize}
%
\end{frame}

\begin{frame}
  \frametitle{Option pricing using interest rate trees}

  At expiration date $T$ the price of the option is equal to
  the payoff for the buyer, i.e.,
  %
  \begin{equation}
    \operatorname{max} \{ 0, S - K \}
  \end{equation}
  %
  thus we know the price of the option at all possible
  states in expiration (less than or equal $2n + 1$).

  Then, for each node at time $t_{n-1}$ we can compute the
  price at one time-step earlier, by discounting the price
  one time-step in the future back (times probabilities).

  (Emphasis on probabilites and link to tree.)

\end{frame}
