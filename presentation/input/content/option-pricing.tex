\section{Part II: Option pricing using interest rate trees}

\newcommand{\tabitem}{~~\llap{\textbullet}~~}


%% options and interest rate trees
\begin{frame}
\frametitle{What are options?}
%
%
\begin{block}{Call options (informally)}
An option is the right but not the obligation to buy an
asset for a price at a time in the future.
\end{block}
%
Define the following:
%
\begin{itemize}
  \item Price of underlying asset: $S$
  \item Strike price: $K$
  \item Maturity date: $T$
\end{itemize}
%
%
\begin{block}{Call options (formally)}
An option is a financial contract between two parties, such
that the buyer has the right but not the obligation to buy
the underlying asset at price $K$, when the contract expires
at time $T$. The time $T$ price can be described as:
\begin{equation}
  C_T = \operatorname{max} \{ 0, S - K \}
\end{equation}
\end{block}
\end{frame}


\begin{frame}
  \frametitle{Interest rates}
%
  \begin{block}{Objective: Tree of short rates}
    Compute tree of short rates given current structure of
    long rates.
  \end{block}

What is the current structure of long rates?
\begin{itemize}
\item Usually quoted as the yields on zero-coupon bonds with
  different (often long) time to maturity.
\end{itemize}
%
\vspace{0.25cm}
\hrule
\vspace{0.25cm}
%
\begin{itemize}
\item Need rates for shorter time periods.
\item Trinomial-tree of interest rates [HW94]. Two-stage
  process:
\begin{enumerate}
  \item Compute tree with nodes equidistant in $r$ and $t$.
  \item Introduce time-varying drift.
\end{enumerate}
%
\end{itemize}
\end{frame}

\begin{frame}
  \frametitle{Interest rates}
  %
\textbf{Step 1:} Compute tree with nodes equidistant in $r$ and $t$.
%
\begin{enumerate}
\item Define $(i, j)$ as the node for which $t = i \Delta t$
  and $r = j \Delta r$.
\item Define boundaries (jmax and jmin).
\item Define probabilites $P_u$, $P_m$, and $P_d$ (depends
  only on $j$).
\end{enumerate}
%
\vspace{0.25cm}
\hrule
\vspace{0.25cm}
%
\textbf{Step 2:} Introduce time-varying drift.
\begin{enumerate}
\item Set $Q_{(0,0)} = 1$ and $\alpha_{0}$ to initial
  $\Delta t$ period interest rate (interpolate if needed).
\item Compute $Q_{(1,1)}, Q_{(1,0)}$ and $Q_{(1,-1)}$.
\item Compute $\alpha_{1}$.
  %
  \begin{itemize}
  \item Discount value of bond from $2\Delta t$ to $\Delta
    t$ (interpolate if needed).
  \item Use $Q$s to discount from $\Delta t$ to 0.
  \item Should equal price of zero maturing at time $2\Delta
    t$: Solve for $\alpha_{1}$.
  \end{itemize}
  %
\item Shift rates at time $i = 1$ by $\alpha_{1}$.
\item Repeat steps 2-4 (closed formulas, dep. on previous values).
\end{enumerate}
%
\end{frame}


\begin{frame}[fragile]
  \frametitle{Option pricing using interest rate trees}
  %
  The time $T$ price of a call option is given by
  \begin{equation}
    C_T = \operatorname{max} \{ 0, S - K \}
  \end{equation}
  i.e, we know all time $T$ option prices.
\vspace{0.25cm}
\hrule
\vspace{0.25cm}
  The present value of any cash flow is found by discounting
  the cash flow back, i.e.,
  %
  \begin{enumerate}
  \item Compute prices at time $t_n = T$ (known).
  \item Discount cash flow back from $t_n$ to $t_{n-1}$,
    using probabilites and short rates (with drift!).
  \item Repeat step 2, i.e., discount cash flow one $\Delta
    t$-period back at a time until you reach $t_0$.
  \item Now you have the current price of the option which
    is consistent with the current market term structure
    (long rates).
  \end{enumerate}
  %
\end{frame}
