\section{Introduction}

The demand for computational power is ever-increasing and as
a consequence, a growing interest in general-purpose
computing on GPUs emerged. The GPU source of power -- a
highly parallel structure -- imposes some challenges, one of
them being writing parallel code. Functional programming has
shown to be a good way of approaching this challenge as it
allows writing data-parallel code using inherently-parallel
operators.

The above reason is the main motivation behind the Futhark
programming language~\cite{futharkweb}: A pure functional
data-parallel array language and a heavily optimizing
compiler which is able to generate parallel code running on
GPUs.

In this document we describe a model developed by Hull and
White~\cite{HullWhite94} for computing interest rate trees,
and how such trees can be used to price derivatives. This
approach has been implemented in Futhark for option pricing
and the task of this project is to translate this program
into CUDA.
%% Due to the financial crisis in 2008 the general demand of
%% computation was further enhanced in the financial sector,
%% .e.g., banks want to perform a vast amount of computations
%% when making risk assessments enabling them to make better
%% decisions.
