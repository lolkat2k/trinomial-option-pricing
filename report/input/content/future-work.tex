\section{Future work}

Unfortunately the program does not validate at the current
time of writing. For future work we should look into how
the first loop calculates the Qs and the alphas.
Right now our alphas become negative, which just seems wrong.
We should take a look at how they are computed from tmpss,
and try and figure out how alpha\_inds and alpha\_vals are
used to update specific alphas. If we get that part right
we are very close to validating as the second loop is really
tiny compared to what we have already implemented.

Despite the lack of an in-depth knowledge of derivatives
pricing in practice we propose two possible optimization
strategies that could be investigated. Both proposals are
based on splitting the described pricing process into two
steps: 1) compute the tree of short rates, and 2) compute
the derivative price.

The first approach is based on partial evaluation and the
less known data specialisation
technique~\cite{malmkjaer,chirokoff}. Partial evaluation
aims at precomputing all static input known at compile time,
generating a new semantically equivalent but faster
program. Data specialisation on the other hand, stores the
precomputed values in a data structure which is then passed
to a modified version of the original program that is able
to use the data structure. Using the latter technique, if some
range of interest dependent derivatives use the same tree of
short rates, we need only to compute this tree once. With
the former technique faster pricing programs could be
generated automatically, say once per hour, by inlining the
short rate tree in the pricing program, if the market term
structure is only updated in intervals of one hour.

The second approach is hinted to in~\cite{HullWhite94} and
mentions how an extension of the described model is able to
fit two term structures simultaneously. It would be worth
investigating if this extension has some underlying general
pattern allowing multiple term structures to be fitted
simultaneously.
