\section{Options and interest rates}

In this section we give a quick introduction to options and
interest rates which we use to value them.

\subsection{Interest rates}

Introduce the interest rates we are using: the short rate,
zero-coupon bonds, and the yield curve. (Maybe also the
interest tree.)

\subsection{What are options?}

We normally distinguish between two styles of options,
namely European and American style options. The following
introduces European style options and they are also the only
products that we price in this paper, but American style
options are straightforwardly introduced in continuation
hereof.

An European call option is a financial contract in which the
buyer of the contract pays the seller of the contract a
premium for the right but not the obligation to buy some
other asset when the contract expires. This other asset,
also called the underlying asset, is specified in the
contract and you can think of it as a single stock but in
reality it can be any kind of asset.

Being a bit more formal, we have that: The buyer and seller
enter into the contract at time $t$, and the contract
expires at time $T$, also called the maturity date. The
payoff at time $T$ for the buyer of the call option is
described as
%
\begin{equation}
  C_T = \operatorname{max}\{0, S_T-K\}
\end{equation}
%
in which $S_T$ is the price at time $T$ of the underlying
asset and $K$ is the price at which the buyer of the call
option is allowed to acquire the underlying asset, also
called the strike price. Note, that the seller of the call
option must own the underlying asset at time $T$ in case
that the buyer decides to buy the asset!

Almost dual to a European call option we have an European
put option, in which the buyer of the put option has the
right but not the obligation to sell the underlying
asset. The payoff at time $T$ for the buyer of the put
option is described as
%
\begin{equation}
  P_T = \operatorname{max}\{0, K-S_T\} .
\end{equation}
%
The put option is a bit more tricky because the terms buyer
and seller is a bit confusing here: Opposite to the call
option, the buyer of the put option owns the underlying
asset. Then at maturity date, if the underlying asset is
worth less than the strike price, the buyer will decide to
sell the underlying asset to the seller of the put
option. By selling for $K$ instead of $S_T$ the buyer of the
put option obtains more value than if she had sold the
underlying asset at the market price $S_T$.

American style options allow the buyer to decide to exercise
her right at any given point in time between time $t$ and
$T$, in contrast to European style options where the buyers
right can only be exercised at time $T$.


\subsection{Option pricing using trinomial trees of interest rates}
