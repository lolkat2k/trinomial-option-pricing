\section{Option pricing using interest rate trees}

In this section we give a brief overview of a concrete
derivatives pricing model \cite{HullWhite94} in which a tree
of short rates is computed and used to price a european
style option. This model is linked to an overall
implementation strategy.

\subsection{Options and interest rates}

The problem at hand is option pricing, a concrete example of
an interest rate dependent derivative. This way of pricing
works by discounting future prices back in time by multiple
small stesps, i.e., we first discount back the price of the
option from time $t_{n}$ to $t_{n-1}$, then from time
$t_{n-1}$ to $t_{n-2}$, and so on until we have the price a
time $t_0$. Furthermore, we describe the development of the
option price by considering three different scenarios: if
the price goes up; if it does not change; and if it goes
down. Over multiple time steps this leads to multiple
possible prices at expiration date.

Before we can even begin the described pricing process we
need to compute the short rates needed to discount between
each pair of consecutive points in time.

The model developed by Hull and White describes a method of
computing a tree of short rates from the current market term
structure of long rates: Given the current market term as
input the model describes an analytic solution to computing
all the short rates in the tree. With this tree of short
rates we are now able to compute any derivate which depends
on the short rate.


\subsection{Overall implementation strategy}

The model described above can be divided into two steps: the
first computing the tree of short rates; and the second
computing the derivative price. The former proceeds in a
forward manner while the latter is an inherently backward
process.

To be able to maximise the degree of paralellism we use a
simple but clever insight that were presented to us: Each
option can have many possible prices depending on the path
through the tree, and the price at any state depends on its
three successor states (except for the prices on maturity
dates). Thus, if we use a one-dimensional iteration space,
i.e., we model all threads in one dimension, and stack the
options on top of each other, we can use two convergence
loops to compute the interest tree and one to compute the
option price, respectively.

In this way each block of threads is able to compute a
collection of options, and we maximise the degree of
parallelism as we cannot do any better due to both the
interest rates and prices being dependent on earlier and
later states, respectively.


%% Because we need to discount prices in one period intervals
%% we cannot directly use the market term structure. Given the
%% current market term structure of long rates we construct a
%% tree of possible future short rates that fits the current
%% market term structure. Now, we are able to discount back the
%% option price from its maturity date till today.


%% \subsection{Options}

%% We normally distinguish between two styles of options,
%% namely european and american style options. The following
%% introduces european style options and they are also the only
%% products that we price in this paper, but american style
%% options are straightforwardly introduced in continuation
%% hereof.

%% A european call option is a financial contract in which the
%% buyer of the contract pays the seller of the contract a
%% premium for the right but not the obligation to buy some
%% other asset when the contract expires. This other asset,
%% also called the underlying asset, is specified in the
%% contract and you can think of it as a single stock but in
%% reality it can be any kind of asset.

%% Being a bit more formal, we have that: The buyer and seller
%% enter into the contract at time $t$, and the contract
%% expires at time $T$, also called the maturity date. The
%% payoff at time $T$ for the buyer of the call option is
%% described as
%% %
%% \begin{equation}
%%   C_T = \operatorname{max}\{0, S_T-K\}
%% \end{equation}
%% %
%% in which $S_T$ is the price at time $T$ of the underlying
%% asset and $K$ is the price at which the buyer of the call
%% option is allowed to acquire the underlying asset, also
%% called the strike price. Note, that the seller of the call
%% option must own the underlying asset at time $T$ in case
%% that the buyer decides to buy the asset!

%% Almost dual to a European call option we have an European
%% put option, in which the buyer of the put option has the
%% right but not the obligation to sell the underlying
%% asset. The payoff at time $T$ for the buyer of the put
%% option is described as
%% %
%% \begin{equation}
%%   P_T = \operatorname{max}\{0, K-S_T\} .
%% \end{equation}
%% %
%% The put option is a bit more tricky because the terms buyer
%% and seller is a bit confusing here: Opposite to the call
%% option, the buyer of the put option owns the underlying
%% asset. Then at maturity date, if the underlying asset is
%% worth less than the strike price, the buyer will decide to
%% sell the underlying asset to the seller of the put
%% option. By selling for $K$ instead of $S_T$ the buyer of the
%% put option obtains more value than if she had sold the
%% underlying asset at the market price $S_T$.

%% American style options allow the buyer to decide to exercise
%% her right at any given point in time between time $t$ and
%% $T$, in contrast to European style options where the buyers
%% right can only be exercised at time $T$.


%% \subsection{Interest rates}

%% The market interest structure of long rates is typically
%% quoted in terms of yields on government issued securities,
%% e.g., zero-coupon bonds.


%% \subsection{Option pricing using interest rate trees}

%% Because we need to discount prices in one period intervals
%% we cannot directly use the market term structure. Given the
%% current market term structure of long rates we construct a
%% tree of possible future short rates that fits the current
%% market term structure. Now, we are able to discount back the
%% option price from its maturity date till today.
